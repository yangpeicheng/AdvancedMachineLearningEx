% !Mode:: "TeX:UTF-8"
\documentclass[a4paper,UTF8]{article}
\usepackage{ctex}
\usepackage[margin=1.25in]{geometry}
\usepackage{color}
\usepackage{graphicx}
\usepackage{amssymb}
\usepackage{amsmath}
\usepackage{amsthm}
%\usepackage[thmmarks, amsmath, thref]{ntheorem}
\theoremstyle{definition}
\newtheorem*{solution}{Solution}
\newtheorem*{prove}{Proof}
\usepackage{multirow}
\usepackage{url}
\usepackage{enumerate}
\renewcommand\refname{参考文献}

%--

%--
\begin{document}
\title{实验1. 度量学习实验报告}
\author{MG1733079,杨佩成,\url{18362903155@163.com}}
\maketitle

\section*{综述}
	在机器学习领域,如何选择合适的度量函数一直是一个重要的问题。因为度量函数的选择依赖于学习任务本身,并且度量学习的好坏会直接影响到算法的性能。
为了解决这一问题,我们可以尝试通过学习得到合适的度量函数。距离度量学习(Distance Metric Learning)的目标是学习得到合适的度量函数,使得在该度量下更容易
找出样本之间潜在的联系,进而提高那些基于相似度的学习器的性能。

\section*{任务1}
	\subsection*{度量函数学习目标}
    通常我们使用的很多距离度量表达式都是固定的,表达式的设置往往依赖于使用者的经验和不断试验,不能根据数据的具体分布自动调整表达式。通过度量学习我们可以直接“学习”出一个合适
的距离度量。度量学习中我们通常使用马氏距离(Mahalanobis distance)
\begin{equation}
    dist^2_{mah}(x_i,x_j)=(x_i-x_j)^TM(x_i-x_j)=\Vert{x_i-x_j}\Vert^2_M ,
\end{equation}
\begin{equation}
    p_{ij}=\frac{exp{(-\Vert{Ax_i-Ax_j}\Vert^2)}}{\sum_{k\not=i}exp{(-\Vert{Ax_i-Ax_k}\Vert^2)}}
\end{equation}
\begin{equation}
    p_i=\sum_{j\in C_i}p_{ij}
\end{equation}
\begin{equation}
    f(A)=\sum_i \sum_{j\in C_i}p_{ij}=\sum_i p_i
\end{equation}
\begin{equation}
    \frac{\partial f}{\partial A}=2A\sum_i(p_i \sum_k p_{ik} x_{ik} x^T_{ik}-\sum_{j \in C_i}p_{ij}x_{ij}x^T_{ij})
\end{equation}
\begin{equation}
    A=A+\alpha \frac{\partial f}{\partial A}
\end{equation}
\noindent其中$M$为“度量矩阵”,而度量学习就是要对$M$进行学习。本次实验
	\subsection*{优化算法}
		\dots	
	
\section*{任务2}
	\dots

\end{document} 